%!TEX TS-program = xelatex
%!TEX encoding = UTF-8 Unicode
% Awesome CV LaTeX Template for Cover Letter
%
% This template has been downloaded from:
% https://github.com/posquit0/Awesome-CV
%
% Authors:
% Claud D. Park <posquit0.bj@gmail.com>
% Lars Richter <mail@ayeks.de>
%
% Template license:
% CC BY-SA 4.0 (https://creativecommons.org/licenses/by-sa/4.0/)
%


%-------------------------------------------------------------------------------
% CONFIGURATIONS
%-------------------------------------------------------------------------------
% A4 paper size by default, use 'letterpaper' for US letter
\documentclass[11pt, a4paper]{awesome-cv}

% Configure page margins with geometry
\geometry{left=1.4cm, top=.8cm, right=1.4cm, bottom=1.2cm, footskip=.5cm}

% Color for highlights
% Awesome Colors: awesome-emerald, awesome-skyblue, awesome-red, awesome-pink, awesome-orange
%                 awesome-nephritis, awesome-concrete, awesome-darknight
\colorlet{awesome}{awesome-red}
% Uncomment if you would like to specify your own color
% \definecolor{awesome}{HTML}{CA63A8}

% Colors for text
% Uncomment if you would like to specify your own color
% \definecolor{darktext}{HTML}{414141}
% \definecolor{text}{HTML}{333333}
% \definecolor{graytext}{HTML}{5D5D5D}
% \definecolor{lighttext}{HTML}{999999}
% \definecolor{sectiondivider}{HTML}{5D5D5D}

% Set false if you don't want to highlight section with awesome color
\setbool{acvSectionColorHighlight}{true}

% If you would like to change the social information separator from a pipe (|) to something else
\renewcommand{\acvHeaderSocialSep}{\quad\textbar\quad}


%-------------------------------------------------------------------------------
%	PERSONAL INFORMATION
%	Comment any of the lines below if they are not required
%-------------------------------------------------------------------------------
% Available options: circle|rectangle,edge/noedge,left/right
\name{Yii}{Kuo Chong}
\position{Software Engineer{\enskip\cdotp\enskip}Typescript Developer}
% \address{235, World Cup buk-ro, Mapo-gu, Seoul, 03936, Republic of Korea}

\mobile{(+60) 14-881-7836}
\email{byu1628@proton.me}
%\dateofbirth{January 1st, 1970}
%\homepage{www.posquit0.com}
\github{gohanko}
\linkedin{gohanko}
% \gitlab{gitlab-id}
% \stackoverflow{SO-id}{SO-name}
% \twitter{@twit}
% \skype{skype-id}
% \reddit{reddit-id}
% \medium{madium-id}
% \kaggle{kaggle-id}
% \hackerrank{hackerrank-id}
% \googlescholar{googlescholar-id}{name-to-display}
%% \firstname and \lastname will be used
% \googlescholar{googlescholar-id}{}
% \extrainfo{extra information}

\quote{``Be the change that you want to see in the world."}


%-------------------------------------------------------------------------------
%	LETTER INFORMATION
%	All of the below lines must be filled out
%-------------------------------------------------------------------------------
% The company being applied to
\recipient
  {Hennge Recruitment Team}
  {Hennge\\Daiwa Shibuya Square,\\16-28 Nanpeidaicho, Shibuya City, Tokyo, 150-0036, Japan}
% The date on the letter, default is the date of compilation
\letterdate{\today}
% The title of the letter
\lettertitle{Job Application for Global Internship Program}
% How the letter is opened
\letteropening{To whom it may concern,}
% How the letter is closed
\letterclosing{Sincerely,}
% Any enclosures with the letter
\letterenclosure[Attached]{Curriculum Vitae}


%-------------------------------------------------------------------------------
\begin{document}

% Print the header with above personal information
% Give optional argument to change alignment(C: center, L: left, R: right)
\makecvheader[R]

% Print the footer with 3 arguments(<left>, <center>, <right>)
% Leave any of these blank if they are not needed
\makecvfooter
  {\today}
  {Yii Kuo Chong~~~·~~~Cover Letter}
  {}

% Print the title with above letter information
\makelettertitle

%-------------------------------------------------------------------------------
%	LETTER CONTENT
%-------------------------------------------------------------------------------
\begin{cvletter}

\lettersection{Introduction}
My name is Yii Kuo Chong, a final year systems engineering student at Universiti Tunku Abdul Rahman and I'm writing to express my interest in joining Hennge's Global Internship Program (Batch 6 2024).

\lettersection{Why Hennge}
I want to find a diverse, and international environment where I can actually learn a lot. To make my internship worth it so to say, and Hennge fits the bill very well.

Through reading the company's blog, I'm impressed at how seriously the company takes it's internship program. It seems that from a couple accounts, interns are require to undergo a training course where they have to build and deploy a fullstack application. 

I believe I can learn a lot from this. In particular, I'm interested in how software is managed and developed, how the technologies is used, and I'm looking to learn some devops knowledge as I never really had the chance to dabble in devops professionally.

\lettersection{Why me?}
I am a professional software developer with experience building scaleable frontend/backend applications in NodeJS/Python for 3 years. Before that, I dabbled in programming since 2016 as a hobbyist. Ocassionally, I work on personal projects, and accept contract work.

In 2021, I decided to pursue a degree in Information Systems Engineering, where I have consistently maintained a 3.0 CGPA since my education began. Not only that, I've developed a fullstack IOT Debugging tool to assist my professor in his research project.

Currently, I'm aiming to gain experience working overseas so I am interested in continue working with Hennge after my internship.

\lettersection{Conclusion}
All in all, I would love to intern at Hennge. Please feel free to reach out to me through email if you have any questions about my qualifications. Thank you for reading my letter. I look forward to hearing from you.

\end{cvletter}

%-------------------------------------------------------------------------------
% Print the signature and enclosures with above letter information
\makeletterclosing

\end{document}